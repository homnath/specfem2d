%------------------------------------------------------------------------------------------------%

\chapter*{Simulation features supported in SPECFEM2D}
\addcontentsline{toc}{chapter}{Simulation features supported in SPECFEM2D}

%------------------------------------------------------------------------------------------------%

The following lists all available features for a SPECFEM2D simulation,
where {\it CPU} and {\it CUDA} denote the code versions for CPU-only simulations and
CUDA hardware support, respectively.
%
\begin{table}[htp]
\vspace{-1cm}
\label{table:features}
\begin{center}
\begin{tabular}{ l l c c}
\hline
%{\bf Feature}    &   & \multicolumn{3}{c}{{\bf Code version}} \\
%\cmidrule(lr){3-5}
%           &     & {\it CPU} & {\it CUDA} \\
%% to have proper title w/ pandoc
{\bf Feature}   &   & {\it CPU} & {\it CUDA} \\
\hline
& & & \\
%%
{\bf Physics}   & P-SV waves                & X  & X \\
                & SH/membrane waves         & X  & X \\
                & Acoustic                  & X  & X \\
                & Elastic                   & X  & X \\
                & Poroelastic               & X  & - \\
                & Electromagnetic           & X  & - \\
                & Anisotropy                & X  & X \\
                & Attenuation               & X  & X \\
\hline
& & & \\
%%
{\bf Simulation Setup}  & Axisymmetric (2.5D) simulations   & X  & - \\
                        & Noise simulations                 & X  & X \\
                        & initial field                     & X  & X \\
                        & C-PML                             & X  & - \\
                        & Periodic boundaries               & X  & - \\
                        & Simultaneous runs                 & X  & X \\
\hline
& & & \\
%%
{\bf Meshing}           & in-house mesher                       & X  & - \\
                        & external (CUBIT/Trelis,Gmsh)          & X  & - \\
                        & SCOTCH/Metis/Analytical partitioning  & X  & - \\
\hline
& & & \\
%%
{\bf Sensitivity kernels} & Undoing of attenuation          & X  & X \\
                          & Anisotropic kernels             & X  & X \\
                          & Isotropic kernels               & X  & X \\
                          & Approximate Hessian             & X  & X \\
\hline
& & & \\
%%
{\bf Time schemes}  & Newmark           & X  & X \\
                    & LDDRK             & X  & - \\
                    & RK4               & X  & - \\
                    & symplectic PEFRL  & X  & - \\
\hline
& & & \\
%%
{\bf Visualization} & JPEG images          & X  & X \\
                    & Postscript snapshots & X  & X \\
                    & Wavefield dumps      & X  & X \\
\hline
& & & \\
%%
{\bf Seismogram formats}  & Ascii          & X  & X \\
                          & Binary         & X  & X \\
                          & SU             & X  & X \\
                          & down-sampling  & X  & X \\
%
\hline
& & & \\ % to avoid clashes with pandoc
\end{tabular}
\end{center}
\end{table}


