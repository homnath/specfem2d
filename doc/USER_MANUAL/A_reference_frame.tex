%------------------------------------------------------------------------------------------------%

\chapter{Reference Frame Convention}\label{cha:Coordinates}

%------------------------------------------------------------------------------------------------%

The code uses the following convention for the Cartesian reference frame:
\begin{itemize}
\item the $x$ axis points right (i.e., East)
\item the $z$ axis points up (i.e., North).
\end{itemize}


%------------------------------------------------------------------------------------------------%
\subsection*{Seismogram outputs}
%------------------------------------------------------------------------------------------------%

The seismogram output directions are given in Cartesian $x$/$z$
directions. They can be rotated (from the direction of positive $z$, i.e. from the North) if needed using a flag defined in the \texttt{Par\_file}.

For the labeling of the seismogram channels, see Appendix~\ref{cha:channel-codes}.
Additionally, we add labels to the synthetics using the following
convention: For a receiver station located in an
\begin{description}
\item [{elastic domain:}] ~
\begin{itemize}
\item \texttt{semd} for the displacement vector
\item \texttt{semv} for the velocity vector
\item \texttt{sema} for the acceleration vector
\end{itemize}
\item [{acoustic domain:}] ~\newline
 (please note that receiver stations in acoustic domains must be buried.
This is due to the free surface condition which enforces a zero pressure
at the surface.)
\begin{itemize}
\item \texttt{semd} for the displacement vector
\item \texttt{semv} for the velocity vector
\item \texttt{sema} for pressure which will be stored in the vertical component
\texttt{Z}. Note that pressure in the acoustic media is isotropic
and thus the pressure record would be the same in the other two component
directions. We therefore use the other two seismogram components to
store the acoustic potential in component \texttt{X} (or \texttt{N})
and the first time derivative of the acoustic potential in component
\texttt{Y} (or \texttt{E}).
\end{itemize}
\end{description}
The seismograms are by default written out in ASCII-format to the
\texttt{OUTPUT\_FILES/} subdirectory by each parallel process.

